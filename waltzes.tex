\newcommand{\TestAllWaltz}{
\BigWaltz % Большой фигурный вальс
}
% \Dance{\command}{\Waltz}{Вальс Пламя свечи}{Вальс <<Пламя свечи>>}{
% dancetext }

\Dance{\BigWaltz}{\Waltz}{Большой фигурный вальс
}{Большой фигурный вальс}{
\Start К спиной к центру круга, Д -- лицом. Кавалер подает даме правую руку, Д -- левую, отведя поданные руки в сторону.

\DanceBeatNCap{
4 тт & балансе по линии танца -- против л.т. Касаются только левая рука дамы и правая К. Повторить фигуру.\\

2 тт & променад по линии танца -- пара раскрывается и закрывается (начиная с внешних ног), делая два вальсовых шага по л.т.\\}
\DanceBeatNCap{
2 тт & расход спинами: К и Д опускают руки и делают два вальсовых шага -- на первый они разворачиваются спиной друг к другу, на второй -- становятся лицом. Д делает поворот по часовой стрелке, К -- против ч. с.\\

8 тт & вальс\\
}

\DanceBeatNCap{
4 тт & раскрытие-закрытие -- 2 раза\\
& Раскрытие -- пара раскрывается по л.т., как в променаде, но чуть сильнее разворачиваясь друг к другу спинами. На закрытии К и Д оказываются лицом друг к другу, но рук не подают.\\

2 тт & променад по л.т.\\

2 тт & расход спинами. В конце -- становимся в «дощечку»: Д, делая второй вальсовый шаг при расходе, чуть не доворачивается до исходного положения. К -- наоборот, делает поворот чуть больше, чем до исходного положения. На начало «дощечки» -- Д стоит спиной по л.т., а К -- лицом. Левая рука у каждого за спиной и держит вытянутую правую руку партнера.}
\DanceBeatNCap{
4 тт & «дощечка»\\
& Положение в паре: Д стоит правым боком к центру, а К -- левым, спиной друг другу. Д стоит у правого плеча К. Левая рука у каждого за спиной и держит вытянутую правую руку партнера.\\

& Четыре вальсовых шага в таком положении по кругу по ч.с. Проходим около круга.\\
}

\DanceBeatNCap{
4 тт & обвод дамы\\
& К левой рукой берет правую руку дамы и обводит ее вокруг себя. В конце фигуры -- К меняет руку на правую, правые руки К и дамы подняты над головой дамы.\\

8 тт & Д идет под рукой вальсовым шагом, К -- вальсовой дорожкой вперед по л.т.}
\DanceBeatNCap{
4 тт & движение «по кругу» -- К берет левой рукой левую руку дамы, правая -- у дамы за спиной, не касаясь ее. Вальсовым шагом проходят небольшой круг против часовой стрелки, заканчивая лицом по л.т.\\

4 тт & 4 па марше (4 шага, по одному шагу на такт, с правых ног), на последнем подают руки: у дамы правая поднята наверх, левая -- как бы обнимает талию. Кавалер берет левой рукой правую руку дамы, а правой -- левую.\\

4 тт & движение по л.т., «вертушка».\\
}
\DanceBeatNCap{
& \textbf{Кавалер}: делает 4 шага вальсовой дорожки по л.т., на четвертом шаге разворачиваясь лицом против л.т.\\

& \textbf{Дама:}\\
1 тт & вальсовый шаг по ч.с., становясь спиной к К\\

2 тт & вальсовый шаг по ч.с, становясь лицом к К. Руки -- лодочкой.\\

3 тт & К пропускает Д под своей правой рукой\\

4 тт & Д заканчивает поворот, оказывается лицом против л.т. Руки приходят в исходное положение, как в начале «вертушки»\\

4 тт & движение против л.т., «вертушка».\\}
\DanceBeatNCap{
& \textbf{Кавалер:} делает 4 шага вальсовой дорожки против л.т., на четвертом шаге разворачиваясь лицом к даме, спиной к центру зала.\\

& \textbf{Дама:}\\

1 тт & вальсовый шаг против л.т. по прямой\\

2 тт & вальсовый шаг против часовой стрелки, становясь спиной к К\\

3 тт & вальсовый шаг против ч.с, становясь лицом к К.\\

4 тт & Д делает поворот под правой рукой К.\\

8 тт & вальс.\\
}
}