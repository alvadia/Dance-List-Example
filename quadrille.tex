\newcommand{\TestAllQuad}{
\Bat % Кадриль Летучая мышь
\FrenchQuadrille % Французская кадриль в каре
}
\newcommand{\loadFQScheme}[2][0.33]{\loadIllustration[#1]{FrenchQuadrille/}{#2}}

%\Dance{\command}{\Quadrille}{Кадриль name}{Кадриль name}{
%dancetext }

% 
\Dance{\FrenchQuadrille}{\Quadrille}{Французская кадриль в каре}{Французская кадриль (Quadrille Française) в каре}{
Схема взята с сайта \href{https://hda.org.ru/dances/quadrille-francaise-frantsuzskaya-kadril/}{Ассоциации исторического танца}.

\Start \loadFQScheme[0.15]{qf-0}

\DanceBeat{1. Le Pantalon. I-II}{
Тт 1-8 & Chaîne anglaise\\

Тт 9-16 &  Balancer, Tour des mains (Поворот за обе руки)\\

Тт 17-24 &  Chaîne des dames\\

Тт 25-32 &  Demi–promenade, Demi–chaîne anglaise\\

& Фигуру повторяют III-IV пары
}

\DanceBeat{2. L’Eté}{

Тт 1-8 &  К1 и Д2: En avant et en arrière (пары сходятся и расходятся),\\
& A Droite et à gauche (вправо-влево)\\
& \loadFQScheme{qf-1}\\

Тт 9-12 &  К1 и Д2: Traverser (смена мест) на противоположные места\\
& \loadFQScheme{qf-2}\\

Тт 13-16 &  К1 и Д2: A Droite et à gauche с некоторым продвижением вперед\\

Тт 17-20 &  К1 и Д2: Traverser или Balancer к своим партнерам,\\

& тем временем К2 и Д1: Balancer\\

Тт 20-24 &  Пары 1 и 2: Tour des mains на своих местах\\
& Фигура повторяется К2 и Д1, К3 и Д4, К4 и Д3\\
}

\DanceBeat{3. La Poule}{

Тт 1-4 &  К1 и Д2: Traverser правыми руками\\
& \loadFQScheme{qf-3}\\

Тт 5-8 &  К1 и Д2: Retraverser левыми руками, подавая правую руку своему партнеру и таким образом образовывая линию:\\
& \loadFQScheme{qf-4}\\

Тт 9-12 &  пары 1 и 2: Balancer en ligne не отпуская рук\\

Тт 13-16 &  пары 1 и 2: Demi–promenade, в конце пара, где дама шла слева от кавалера, доворачивается до правильных позиций\\

Тт 17-20 &  К1 и Д2: En avant et en arrière\\
& \loadFQScheme{qf-5}\\

Тт 21-24 &  К1 и Д2: Dos à dos\\
& \loadFQScheme{qf-6}\\

Тт 25-32 &  пары 1 и 2: En avant et en arrière (вперёд-назад), Demi–chaîne anglaise\\
& Фигуру повторяют К2 и Д1, К3 и Д4, К4 и Д3\\
}

\DanceBeat{4. La Trenise}{

Тт 1-8 &  Chaîne des dames\\

Тт 9-16 &  Balancer, Tour des mains\\

Тт 17-24 &  К1 и Д1: En avant et en arrière, после этого К отводит даму к К2, оставляет ее слева от него и возвращается на свое место\\

Тт 25-28 & 
Д1 и Д2: заходят за спину К1, и меняются местами соответственно на позиции К1 и Д1\\
& К1: En Avant между дамами и Balancer к ним\\
& \loadFQScheme{qf-7}\\

Тт 29-32 &  К1 на свое место и Balancer, Д1 и Д2: повторяют тт 25-28, при этом расходятся на свои места за К1 и становятся с разных сторон К2\\

Тт 32-40 &  Balancer, Tour des mains, при этом К1 и Д1 на Balancer сходятся и на Tour des mains возвращаются на свои исходные места\\
}
\DanceBeat{5. La Finale}{
Тт 1-8 &  Все вместе Chassé croisé\\
Тт 9-32 &  Фигура L’Eté\\
Тт 33-40 &  Все вместе Chassé croisé\\
}
}

\Dance{\Bat}{\Quadrille}{Кадриль Летучая мышь}{Кадриль Летучая мышь}{
\DancePart{Поклоны:}{
1-2 & К и Д поворачиваются к своему партнеру\\
3-4 & Одновременно К кланяется Д, и Д отвечает реверансом\\
5-6 & К поворачивается к Д, стоящей от него слева, Д к К справа\\
7-8 & К кланяется чужой Д, та отвечает реверансом\\
}
\DancePart{I фигура:}{
1-8 & Полный сhaîne anglaise\\
9-16 & Полный Promenade в положении <<корзиночка>>\\
17-24 & Сhaîne des dames\\
& Повтор фигуры для II и IV пар.
}

\DancePart{II фигура:}{
1 & К и Д разворачиваются друг к другу\\

2-4 & Все делают 3 па-де-баска (из круга, в круг, из круга)\\

5-8 & Круг за обе руки со своими партнерами\\

9-10 & Д подходят к К, стоящим справа от них. К берет подошедшую Д за руку\\

11-12 & Все идут в круг, в кругу К отпускает руку Д слева и берет руку своей Д\\

13-14 & Все идут из круга до своих мест\\

15-16 & Партнеры подают друг другу правые руки и меняются местами по часовой стрелке\\

17-24 & Повторение предыдущей части, но теперь кавалеры уходят направо\\

25-28 & К разворачиваются влево, Д вправо, скрещивают правые предплечья со стоящим там партнером и делают с ним полный оборот по часовой стрелке\\

29-32 & Партнеры скрещивают левые предплечья в своих парах и делают полный оборот до своих мест\\
}
\DancePart{III фигура:}{

& Третья фигура начинается с поклонов. На последних счетах К выводит Д перед собой так, чтобы она стояла спиной в круг\\

1-2 & Все по кругу делают три приставных шага вправо, проходя четверть круга\\

3-4 & Все подают руки оказавшемуся перед ним партнеру визави и меняются с ним местами по часовой стрелке\\

5-8 & Повторение 1-4, но теперь приставные шаги делаются влево\\

9-16 & Повторение 1-8, последний разворот завершается выходом на исходную позицию\\
}
\DancePart{IV фигура:}{

1-2 & I и III пары сходятся в центре и на последний счет разворачиваются лицом к партнёрам\\

3-4 & I и III пары делают два па-де-баска (в круг, из круга)\\

5-8 & К и Д I и III пар разворачиваются на 180 через центр кадрили, проходят из круга, затем проходит до своего места по дуге круга. В это же время К и Д I и IV пар разворачиваются от своего партнера, проходят до места I и III, затем разворачиваются в круг, проходят в центр, встречаются там со своим партнером и проходят с ним до своих мест\\

1-8 & Мулине дам правыми руками, затем левыми\\

& Затем эту фигуру повторяют II и IV пары, причем в этот раз мулине исполняют К\\
}
\DancePart{V фигура:}{

1-8 & Все пары берутся за руки в круг и идут полный круг влево до своих мест\\

9-16 & I и III пара встают в вальсовую позицию и делают 7 шагов галопа, расходясь спинами К. Затем за 7 шагов галопа возвращаются в исходную позицию\\

17-20 & К1 и К3 берут своих Д правой рукой за талию, левой рукой, поддерживая левую руку Д перед собой, подходят лицом к паре, стоящей от них справа, и спиной отходят на место пары визави\\

21-24 & Половина chaîne anglaise\\

& Повтор для II и IV пар\\
}
\DancePart{Финал:}{

1-2 & К1 и К3 разворачиваются лицом к партнеру, в то время как пары II и IV сходятся в средние кадрили так, чтобы получилось две шеренги по четыре человека\\

3-4 & Все отходят назад\\

5-6 & Все идут вперед и проходят друг мимо друга, расходясь левым плечом\\

7-8 & Продолжая движение, все разворачиваются правым плечом вперед\\

9-12 & В двух четверках (первая -- К1 Д1 Д2 Д4, вторая -- К3 Д3 Д4 К2) мулине правыми руками\\

13-16 & Мулине левыми руками, заканчивая на местах на момент такта 9\\

17-18 & Все идут вперед и проходят друг мимо друга, обходя слева\\

19-20 & Продолжая движение, все разворачиваются правым плечом вперед\\

21-22 & Все идут вперед\\

23-24 & I и III пары берутся за обе руки и делают полный оборот по часовой стрелке до исходной позиции. Во II и IV парах К берет правой рукой левую руку Д и они отходят спиной до исходной позиции\\

& Повтор фигуры для других пар\\
}
}